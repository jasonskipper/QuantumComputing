\documentclass{article}
\usepackage[utf8]{inputenc}
\usepackage{amsmath}

\def\ket#1{|#1\rangle}
\def\bra#1{\langle#1|}

\title{Quantum Computing homework 3}
\author{Michael Skipper, n01162792}
\date{Due November 2, 2020}

\begin{document}

\maketitle

\section{Question 1}
\\ \\
\textbf{Part A}

\begin{align*}
    & X_1 \text{ is initialized to } \frac{1}{7}\ket{0} + \frac{6}{7}\ket1 \text{, and } X_2 \text{ is } \ket{0}
    \\ \\ 
    & CNOT(X_1, X_2) \text{ yields } \frac{1}{7}\ket{00} + \frac{6}{7}\ket{11} 
    \\ \\ 
    & CCOIN_\frac{1}{7} \text{ now yields } \frac{1}{7}\ket{00} + \frac{6}{7}(\frac{1}{7}\ket{10} + \frac{6}{7}\ket{11})
    \\ \\ 
    & \text{The joint state so far is } \frac{1}{7}\ket{000} + \frac{6}{49}\ket{100} + \frac{36}{49}\ket{110}
    \\ \\ 
    & CCNOT(X_1, X_2, X_3) \text{ now yields } \frac{1}{7}\ket{000} + \frac{6}{49}\ket{100} + \frac{36}{49}\ket{111}
    \\ \\ 
    & CCOIN_\frac{1}{7}(X_1) \text{ now yields } \frac{1}{7}\ket{000} +
    \frac{6}{49}(\frac{1}{7}\ket{000} + \frac{6}{7}\ket{100}) + \frac{36}{49}(\frac{1}{7}\ket{011} + \frac{6}{7}\ket{111})
    \\ \\
    & \text{So, the joint state } (X_1, X_2, X_3) \text{ is: } 
    \\ \\
    &\frac{1}{7}\ket{000} + \frac{6}{49\times{7}}\ket{000} + \frac{6\times{6}}{49\times{7}}\ket{100}
    + \frac{36}{49\times{7}}\ket{011} + \frac{6\times{36}}{49\times{7}}\ket{111}
    \\ \\
    &\frac{55}{343}\ket{000} + \frac{36}{343}\ket{100} + \frac{36}{343}\ket{011} + \frac{216}{343}\ket{111}
    \\ \\
\end{align*}
\textbf{Part B}
\begin{align*}
    &\ket{0}: \frac{55}{343} + \frac{36}{343} = \frac{91}{343}
    \\ \\
    &\ket{1}: \frac{36}{343} + \frac{216}{343} = \frac{252}{343}
\end{align*}

\textbf{Part C} 
\begin{align*}
    \ket{0}:& \\ 
    &\frac{55}{343}\ket{000} + \frac{36}{343}\ket{011}
    \\ \\ 
    &\frac{\frac{55}{343}}{\frac{91}{343}}\ket{000} + \frac{\frac{36}{343}}{\frac{91}{343}}\ket{011}
    \\ \\ 
    &\frac{55}{91}\ket{000} + \frac{36}{91}\ket{011}
    \\ \\ \\ \\ 
    \ket{1}:& \\ 
    &\frac{36}{252}\ket{100} + \frac{216}{252}\ket{111}
\end{align*}

\section{Question 2}
\textbf{Part A}
\begin{align*}
    & X_1 \text{ initialized to } \ket{1}, X_2 \text{ is } \ket{1}, X_3 is \ket{0}
    \\ \\ 
    & CNOT(X_2, X_3) \text{ yields } \ket{11}
    \\ \\ 
    & H(X_2) \text{ now yields } \frac{1}{\sqrt{2}}\ket{11} + \frac{1}{\sqrt{2}}\ket{10}
    \\ \\ 
    &\text{ joint state } (X_1, X_2, X_3) \text{ so far is } \frac{1}{\sqrt{2}}\ket{110} + \frac{1}{\sqrt{2}}\ket{100}
    \\ \\ 
    & CCNOT(X_1, X_2, X_3) \text{ now yields } \frac{1}{\sqrt{2}}\ket{111} + \frac{1}{\sqrt{2}}\ket{100}
    \\ \\ 
    & H(X_3) \text{ now yields } \frac{1}{\sqrt{2}}(\frac{1}{\sqrt{2}}\ket{111} + \frac{1}{\sqrt{2}}\ket{101}) + \frac{1}{\sqrt{2}}(\frac{1}{\sqrt{2}}\ket{011} - \frac{1}{\sqrt{2}}\ket{100})
\end{align*}

\textbf{Part B}
\begin{align*}
& \ket{0}: (\frac{1}{2})^2 + (\frac{1}{2})^2 = \frac{1}{2}
\\ \\ 
& \ket{1}: (\frac{1}{2})^2 + (\frac{1}{2})^2 = \frac{1}{2}
\end{align*}

\textbf{Part C} 
\begin{align*}
    & \ket{0}: \frac{1}{\sqrt{2}}\ket{111} + \frac{1}{\sqrt{2}}\ket{101}
    \\ \\ 
    & \ket{1}: \frac{1}{\sqrt{2}}\ket{011} + \frac{1}{\sqrt{2}}\ket{100}
\end{align*}


\end{document}
